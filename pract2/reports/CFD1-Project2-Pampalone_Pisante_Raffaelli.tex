\documentclass{article}
\usepackage{amsmath}
\usepackage{titlesec}
\usepackage{graphicx}
\usepackage[margin=1in]{geometry}
\usepackage{hyperref}

% Title, date, and author
\title{Project 1}
\author{Your Name, Collaborator's Name}
\date{\today}

\titleformat{\section}
  {\normalfont\normalsize\bfseries} % Format: font style, size, and weight
  {\thesection}{1em} % Label format and spacing
  {}
  \renewcommand{\thesubsection}{\thesection.\alph{subsection}}

\titleformat{\subsection}
  {\normalfont\small\bfseries} % Format: font style, size, and weight
  {\thesubsection}{1em} % Label format and spacing
  {}
\titleformat{\subsubsection}
  {\normalfont\small\bfseries} % Format: font style, size, and weight
  {\thesubsubsection}{1em} % Label format and spacing
  {}

\begin{document}
\begin{titlepage}
    \centering
    \vspace*{1in}
    
    {\Huge\bfseries Project 1\par}
    \vspace{1.5cm}
    {\Large \today\par}
    \vspace{1.5cm}
    {\Large\itshape Antonio Pampalone 23586519 \\ Giuseppe Pisante 23610012\\ Martina Raffaelli 23616907 \par}
    
    \vfill
    \includegraphics[width=0.3\textwidth]{FAU-Logo.png}\par\vspace{1cm} % Adjust the width as needed
   
\end{titlepage}

\newpage
\small

\section*{\Large Task 2.0:}
The momentum equation is a parabolic equation since, if we compute the $\Delta = B^2 - 4AC$ from the general form of the partial differential equations: 
\[
\]
\begin{equation}
  A \frac{\partial^2 u}{\partial x^2} + B \frac{\partial^2 u}{\partial x \partial y} + C \frac{\partial^2 u}{\partial y^2} + D \frac{\partial u}{\partial x} + E \frac{\partial u}{\partial y} + F u + G = 0
\end{equation}

we get $\Delta = 0$, which means that the equation is parabolic.

We now define the additional boundary conditions:
\begin{itemize}
  \item $u(0,y) = 1$
  \item $u(1,y) = 1$
  \item $v(0,y) = 0$
  \item $v(1,y) = 0$
  \item $v(x,\infty) = 0$
\end{itemize}

\section*{\Large Task 2.1:}
The discretization applied to the x-momentum equation takes into account the parabolic nature of the equation. The discretization is performed using the central difference scheme for the y-direction and the backward difference scheme for the x-direction. The discretized equation is as follows:
\[
  u_{i,j}\frac{u_{i,j} - u_{i-1,j}}{\Delta x}  + v_{i,j} \frac{u_{i,j+1} - u_{i,j-1}}{2 \Delta y} = \frac{1}{Re} \frac{u_{i,j+1} - 2u_{i,j} + u_{i,j-1}}{\Delta y^2}
\]

Such equation is a steady-state equation and thus it doesn't require any time-stepping algorithm. 
For this reason, the costraints for convergence are only applied on the spatial discretization.
\section*{\Large Task 2.2:}
This system is solved using a GMRES iterative solver, which allows us to solve the system for a non-symmetric A and the non linearities of the x-momentum.
The solver updates the values of \( u \) and \( v \) at each iteration, denoted as \( u^{(k+1)} \) and \( v^{(k+1)} \), by utilizing the values from the previous iteration, \( u^{(k)} \) and \( v^{(k)} \), to handle the non-linear terms effectively.
The system of partial differential equations is:
The continuity equation is discretized using the backward difference scheme (BDS) along the x-direction and the central difference scheme (CDS) along the y-direction. The discretized continuity equation is given by:
\[
\begin{cases}
  \frac{u_{i,j} - u_{i-1,j}}{\Delta x} + \frac{v_{i,j+1} - v_{i,j-1}}{2 \Delta y} = 0 \\
  u_{i,j}\frac{u_{i,j} - u_{i-1,j}}{\Delta x}  + v_{i,j} \frac{u_{i,j+1} - u_{i,j-1}}{2 \Delta y} = \frac{1}{Re} \frac{u_{i,j+1} - 2u_{i,j} + u_{i,j-1}}{\Delta y^2}
\end{cases}
\]

To solve this, we construct a linear system of equations in the form \( \mathbf{A} \mathbf{x} = \mathbf{b} \), where \( \mathbf{A} \) is a \( 2n^2 \times 2n^2 \) matrix, \( \mathbf{b} \) is a \( 2n^2 \) vector, and \( \mathbf{x} \) is also a \( 2n^2 \) vector. The structure of \( \mathbf{x} \) is as follows:

\[
\mathbf{x} = \begin{bmatrix}
u_{1,1} \\
u_{1,2} \\
\vdots \\
u_{n,n} \\
v_{1,1} \\
v_{1,2} \\
\vdots \\
v_{n,n}
\end{bmatrix}
\]

The structure of the matrix \( \mathbf{A} \) is as follows:
\[
A = 
\begin{bmatrix}
\frac{1}{\Delta x} & \cdots & -\frac{1}{\Delta x} & \cdots & 0 & \cdots & 0 & \cdots \\
\cdots & \frac{u_{i,j}}{\Delta x} - \frac{2}{Re} \left(\frac{1}{\Delta x^2} + \frac{1}{\Delta y^2}\right) & \cdots & \frac{1}{Re \Delta x^2} & \cdots & 0 & \cdots & 0 & \cdots \\
\cdots & \frac{u_{i,j}}{\Delta x} & \cdots & \frac{1}{Re \Delta x^2} & \cdots & 0 & \cdots & 0 & \cdots \\
0 & \cdots & 0 & \cdots & \frac{1}{\Delta y} & \cdots & -\frac{1}{\Delta y} & \cdots \\
0 & \cdots & 0 & \cdots & \frac{v_{i,j}}{2\Delta y} & \cdots & -\frac{2}{Re \Delta y^2} & \cdots \\
0 & \cdots & 0 & \cdots & \frac{v_{i,j}}{2\Delta y} & \cdots & -\frac{2}{Re \Delta y^2} & \cdots
\end{bmatrix}
\]
In addition, the boundary conditions are applied to the matrix \( \mathbf{A} \) and the vector \( \mathbf{b} \) to solve the system of equations.
The boundary conditions are applied to the cells of \( \mathbf{b} \) and \( \mathbf{A} \) as follows:

\begin{enumerate}
  \item For \( u(0,y) = 1 \):
  \begin{itemize}
    \item Set \( b_{(j-1)n+1} = 1 \)
    \item Set \( A_{(j-1)n+1, (j-1)n+1} = 1 \)
  \end{itemize}
  
  \item For \( u(1,y) = 1 \):
  \begin{itemize}
    \item Set \( b_{(j-1)n+n} = 1 \)
    \item Set \( A_{(j-1)n+n, (j-1)n+n} = 1 \) 
  \end{itemize}
  
  \item For \( v(0,y) = 0 \):
  \begin{itemize}
    \item Set \( b_{n^2+(j-1)n+1} = 0 \)
    \item Set \( A_{n^2+(j-1)n+1, n^2+(j-1)n+1} = 1 \)
  \end{itemize}
  
  
   \item For \( v(1,y) = 0 \):
    \begin{itemize}
      \item Set \( b_{n^2+(j-1)n+n} = 0 \)
      \item Set \( A_{n^2+(j-1)n+n, n^2+(j-1)n+n} = 1 \)
    \end{itemize}
  
  \item For \( v(x,\infty) = 0 \):
  \begin{itemize}
    \item Set \( b_{n^2+(n-1)n+i} = 0 \)
    \item Set \( A_{n^2+(n-1)n+i, n^2+(n-1)n+i} = 1 \) 
  \end{itemize}
\end{enumerate}

\section*{\Large Task 2.3:}
import numpy as np \\
import scipy.sparse.linalg as spla \\
M2 = spla.spilu(A)  \\
x = spla.gmres(A,b,M=M2)  \\

HINT per chatty: come utilizzare gmres per risolvere le non linearita del problema, come specificato sopra.

\begin{thebibliography}{9}
    \bibitem{GitHubRepo}
    \textit{CFD Repository},\\
    Available at: \url{https://github.com/GiuseppePisante/CFD.git}
    
    \bibitem{GitHubCopilot}
    \textit{GitHub Copilot},\\
    GitHub. Available at: \url{https://github.com/features/copilot}
    \end{thebibliography}

\end{document}