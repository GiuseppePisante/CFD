\documentclass{article}
\usepackage{amsmath}
\usepackage{titlesec}
\usepackage{graphicx}
\usepackage[margin=1in]{geometry}
\usepackage{hyperref}
\usepackage{amssymb}

% Title, date, and author
\title{Project 1}
\author{Your Name, Collaborator's Name}
\date{\today}

\titleformat{\section}
  {\normalfont\normalsize\bfseries} % Format: font style, size, and weight
  {\thesection}{1em} % Label format and spacing
  {}
  \renewcommand{\thesubsection}{\thesection.\alph{subsection}}

\titleformat{\subsection}
  {\normalfont\small\bfseries} % Format: font style, size, and weight
  {\thesubsection}{1em} % Label format and spacing
  {}
\titleformat{\subsubsection}
  {\normalfont\small\bfseries} % Format: font style, size, and weight
  {\thesubsubsection}{1em} % Label format and spacing
  {}

\begin{document}
\begin{titlepage}
    \centering
    \vspace*{1in}
    
    {\Huge\bfseries Project 1\par}
    \vspace{1.5cm}
    {\Large \today\par}
    \vspace{1.5cm}
    {\Large\itshape Antonio Pampalone 23586519 \\ Giuseppe Pisante 23610012\\ Martina Raffaelli 23616907 \par}
    
    \vfill
    \includegraphics[width=0.3\textwidth]{FAU-Logo.png}\par\vspace{1cm} % Adjust the width as needed
   
\end{titlepage}

\newpage
\small

\section*{\Large Task 2.0:}
The momentum equation is a parabolic equation since, if we compute the $\Delta = B^2 - 4AC$ from the general form of the partial differential equations: 
\[
\]
\begin{equation}
  A \frac{\partial^2 u}{\partial x^2} + B \frac{\partial^2 u}{\partial x \partial y} + C \frac{\partial^2 u}{\partial y^2} + D \frac{\partial u}{\partial x} + E \frac{\partial u}{\partial y} + F u + G = 0
\end{equation}

we get $\Delta = 0$, which means that the equation is parabolic.

We now define the additional boundary conditions:
\begin{itemize}
  \item $u(0,y) = 1$
  \item $v(0,y) = 0$
\end{itemize}

For x=1, we do not need boundary conditions, since it would overconstrain the problem.

\section*{\Large Task 2.1:}
The discretization applied to the x-momentum equation takes into account the parabolic nature of the equation. The discretization is performed using the central difference scheme for the y-direction, to capture the elliptic character of the diffusion process, and the backward difference scheme for the x-direction to better capture the convective term. The discretized equation is as follows:
\[
  u_{i,j}\frac{u_{i,j} - u_{i,j-1}}{\Delta x}  + v_{i,j} \frac{u_{i+1,j} - u_{i-1,j}}{2 \Delta y} = \frac{1}{Re} \frac{u_{i+1,j} - 2u_{i,j} + u_{i-1,j}}{\Delta y^2}
\]

Such equation is a steady-state equation and thus it doesn't require any time-stepping algorithm. 
For this reason, the costraints for convergence are only applied on the spatial discretization.
\\
The discretization of the first derivative in the \(x\)-direction using the Backward Difference Scheme (BDS) is given by:

\[
\frac{\partial u}{\partial x} \approx \frac{u_{i,j} - u_{i,j-1}}{\Delta x}
\]

Using Taylor series expansions for \( u(x) \) around \( x_i \), we have:

\[
u_{i,j-1} = u(x_i - \Delta x) = u(x_i) - \Delta x \frac{\partial u}{\partial x} + \frac{\Delta x^2}{2} \frac{\partial^2 u}{\partial x^2} + O(\Delta x^3)
\]

Therefore, the approximation for the derivative becomes:

\[
\frac{u_{i,j} - u_{i,j-1}}{\Delta x} = \frac{\partial u}{\partial x} + \frac{\Delta x}{2} \frac{\partial^2 u}{\partial x^2} + O(\Delta x^2)
\]

The truncation error for the Backward Difference Scheme is:

\[
T_{\text{BDS}} = -\frac{\Delta x}{2} \frac{\partial^2 u}{\partial x^2} + O(\Delta x^2)
\]

The discretization of the second derivative in the \(y\)-direction using the Central Difference Scheme (CDS) is given by:

\[
\frac{\partial^2 u}{\partial y^2} \approx \frac{u_{i+1,j} - 2u_{i,j} + u_{i+1,j}}{\Delta y^2}
\]

Using Taylor series expansions for \( u(y) \) around \( y_j \), we have:

\[
u_{i+1,j} = u(y_j + \Delta y) = u(y_j) + \Delta y \frac{\partial u}{\partial y} + \frac{\Delta y^2}{2} \frac{\partial^2 u}{\partial y^2} + \frac{\Delta y^3}{6} \frac{\partial^3 u}{\partial y^3} + O(\Delta y^4)
\]

\[
u_{i-1,j} = u(y_j - \Delta y) = u(y_j) - \Delta y \frac{\partial u}{\partial y} + \frac{\Delta y^2}{2} \frac{\partial^2 u}{\partial y^2} - \frac{\Delta y^3}{6} \frac{\partial^3 u}{\partial y^3} + O(\Delta y^4)
\]

Subtracting \( 2u_{i,j} \) from the sum of \( u_{i+1,j} \) and \( u_{i-1,j} \), we get:

\[
u_{i+1,j} - 2u_{i,j} + u_{i-1,j} = 2 \cdot \frac{\Delta y^2}{2} \frac{\partial^2 u}{\partial y^2} + O(\Delta y^4)
\]

Thus, the discretized second derivative becomes:

\[
\frac{u_{i+1,j} - 2u_{i,j} + u_{i-1,j}}{\Delta y^2} = \frac{\partial^2 u}{\partial y^2} + O(\Delta y^2)
\]

The leading truncation error for the Central Difference Scheme is:

\[
T_{\text{CDS}} = \frac{\Delta y^2}{6} \frac{\partial^4 u}{\partial y^4} + O(\Delta y^4)
\]
\subsection*{Summary of Truncation Errors}

Thus, the sum of the truncation errors is:
\[
T_{\text{total}} = T_{\text{BDS}} + T_{\text{CDS}} = -\frac{\Delta x}{2} \frac{\partial^2 u}{\partial x^2} + \frac{\Delta y^2}{6} \frac{\partial^4 u}{\partial y^4} + O(\Delta x^2) + O(\Delta y^4)
\]

\section*{\Large Task 2.2:}
This system is solved using a BiCGStab iterative solver, which allows us to solve the system for a non-symmetric A and the non linearities of the x-momentum.
The solver updates the values of \( u \) and \( v \) at each iteration, denoted as \( u^{(k+1)} \) and \( v^{(k+1)} \), by utilizing the values from the previous iteration, \( u^{(k)} \) and \( v^{(k)} \), to handle the non-linear terms effectively.
The system of partial differential equations is:
The continuity equation is discretized using the backward difference scheme (BDS) along the x-direction and the central difference scheme (CDS) along the y-direction. The discretized continuity equation is given by:
\[
\begin{cases}
  \frac{u_{i,j} - u_{i,j-1}}{\Delta x} + \frac{v_{i+1,j} - v_{i-1,j}}{2 \Delta y} = 0 \\
  u_{i,j}\frac{u_{i,j} - u_{i,j-1}}{\Delta x}  + v_{i,j} \frac{u_{i+1,j} - u_{i-1,j}}{2 \Delta y} = \frac{1}{Re} \frac{u_{i+1,j} - 2u_{i,j} + u_{i-1,j}}{\Delta y^2}
\end{cases}
\]


It is important to note that at the boundaries, we cannot use this discretization directly. Depending on the boundary we have to apply different schemes:
\begin{enumerate}
  \item Left boundary (\( x = 0 \)): we cannot apply BDS on the x-direction. We thus use FDS on such boundary:
  \[
\begin{cases}
  \frac{u_{i,j+1} - u_{i,j}}{\Delta x} + \frac{v_{i+1,j} - v_{i-1,j}}{2 \Delta y} = 0 \\
  u_{i,j}\frac{u_{i,j+1} - u_{i,j}}{\Delta x}  + v_{i,j} \frac{u_{i+1,j} - u_{i-1,j}}{2 \Delta y} = \frac{1}{Re} \frac{u_{i+1,j} - 2u_{i,j} + u_{i-1,j}}{\Delta y^2}
\end{cases}
\]
\item Bottom boundary (\( y = 0 \)): CDS cannot be applied on the y-direction. We thus use FDS on such boundary:
\[
  \begin{cases}
    \frac{u_{i,j} - u_{i,j-1}}{\Delta x} + \frac{v_{i,j} - v_{i-1,j}}{\Delta y} = 0 \\
    u_{i,j}\frac{u_{i,j} - u_{i,j-1}}{\Delta x}  + v_{i,j} \frac{u_{i,j} - u_{i-1,j}}{\Delta y} = \frac{1}{Re} \frac{u_{i-2,j} - 2u_{i-1,j} + u_{i,j}}{\Delta y^2}
  \end{cases}
\]
\end{enumerate}
Since we only apply Dirichlet Boundary Conditions, a different discretization at the boundaries is not relevant for solving the problem. But for generality, we decided to add this information.
To solve this, we first tried to construct a linear system of equations in the form \( \mathbf{A} \mathbf{x} = \mathbf{b} \), where \( A \in \mathbb{R}^{2n^2 \times 2n^2} \), \( b \in \mathbb{R}^{2n^2} \), and \( x \in \mathbb{R}^{2n^2 \times 2n^2} \). The structure of \( \mathbf{x} \) is as follows:

\[
\mathbf{x} = \begin{bmatrix}
u_{1,1} \\
u_{1,2} \\
\vdots \\
u_{n,n} \\
v_{1,1} \\
v_{1,2} \\
\vdots \\
v_{n,n}
\end{bmatrix}
\]

The structure of the matrix \( \mathbf{A} \) is as follows:
\[
  \hspace{-2.5cm}
A = 
\begin{bmatrix}
  -\frac{1}{\Delta x} & \frac{1}{\Delta x} &  &  &  &  &  \vdots &  -\frac{1}{\Delta y} \cdots& \frac{1}{\Delta y} \cdots&  \\
 & \ddots  &  &  &  &  &\vdots &\ddots &  &  \\
 \cdots-\frac{1}{\Delta x} & \frac{1}{\Delta x} &  &  & & &\vdots  &\cdots -\frac{1}{2\Delta y} & &\frac{1}{2\Delta y} \cdots  \\
 & \vdots  &\ddots  &  &  &  &\vdots & &  &  \\
 &  &  &  &  &  &\vdots & &  \cdots -\frac{1}{\Delta y} & \frac{1}{\Delta y}  \\
 \cdots&\cdots  &\cdots  &\cdots  &\cdots  &\cdots  &\cdots &\cdots  &\cdots  &\cdots  \\
 -\frac{v_{i,j}^{(k)}}{\Delta y} + \frac{1}{Re \Delta y^2} - \frac{u_{i,j}^{(k)}}{\Delta x} & \frac{u_{i,j}^{(k)}}{\Delta x} &\cdots \frac{v_{i,j}^{(k)}}{\Delta y} + \frac{2}{Re \Delta y^2} & \frac{1}{Re \Delta y^2} & \cdots  &  &\vdots &  &  &  \\
 \cdots &  -\frac{1}{Re \Delta y^2}\cdots &-\frac{v_{i,j}^{(k)}}{\Delta y} + \frac{2}{Re \Delta y^2} \cdots & -\frac{u_{i,j}^{(k)}}{\Delta x}  & \frac{v_{i,j}^{(k)}}{\Delta y} - \frac{1}{Re \Delta y^2} + \frac{u_{i,j}^{(k)}}{\Delta x}  & \cdots &\vdots & & &  \\
 &\vdots&\vdots&\vdots&\vdots&\vdots&\vdots & &  &  \\
 &\cdots  & \cdots-\frac{v_{i,j}^{(k)}}{2\Delta y} + \frac{1}{Re \Delta y^2} \cdots &  -\frac{u_{i,j}^{(k)}}{\Delta x}& \frac{u_{i,j}^{(k)}}{\Delta x} - \frac{2}{Re \Delta y^2} &\cdots \frac{v_{i,j}^{(k)}}{2\Delta y} + \frac{1}{Re \Delta y^2} \cdots&\vdots &  &  &  \\
 &\vdots&\vdots&\vdots&\vdots&\vdots&\vdots & &  &  \\
 &\cdots   \frac{1}{Re \Delta y^2}\cdots & \frac{v_{i,j}^{(k)}}{\Delta y} + \frac{1}{Re \Delta y^2}\cdots & -\frac{u_{i,j}^{(k)}}{\Delta x} & \frac{u_{i,j}^{(k)}}{\Delta x} - \frac{1}{Re \Delta y^2} -\frac{v_{i,j}^{(k)}}{\Delta y} & \cdots\frac{1}{Re \Delta y^2} &\vdots &  &  & 
 

\end{bmatrix}
\]

However, given the complexity of the system, we were unable to solve it through such method and we thus decided to move towards a different route.
In particular, given the physical nature of the problem, we decided to trat the x spatial variable as a temporal variable, in order to solve the problem in a 1D environment using Explicit Euler.
This was in fact a very good way to solve the problem, as it is clearly more simple and doesn't require the construction of a complex system of equations.
\\
In the context of Explicit Euler, the left boundary condition was thus treated as an initial condition.
\section*{Python Code}
Our Python implementation can be found at the following GitHub repository \cite{GitHubRepo}. In the src folder there can be found two main scripts and two solvers.
One script is related to the 1D solver using Explicit Euler, which works correctly. The other script pertains to the 2D solver, which is still a work in progress. 
The solutions for both \( u \) and \( v \) are included in this report, as requested.
\begin{figure}[h!]
  \centering
  \includegraphics[width=0.8\textwidth]{velocity_distribution.png}
  \caption{Velocity distribution along the domain}
  \label{fig:velocity_distribution}
\end{figure}

\begin{thebibliography}{9}
    \bibitem{GitHubRepo}
    \textit{CFD Repository},\\
    Available at: \url{https://github.com/GiuseppePisante/CFD.git}
    
    \bibitem{GitHubCopilot}
    \textit{GitHub Copilot},\\
    GitHub. Available at: \url{https://github.com/features/copilot}
    \end{thebibliography}

\end{document}