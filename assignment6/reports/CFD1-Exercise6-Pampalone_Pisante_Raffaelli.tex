\documentclass{article}
\usepackage{amsmath}
\usepackage{titlesec}
\usepackage{graphicx}
\usepackage[margin=1in]{geometry}
\usepackage{hyperref}
\usepackage{float}

% Title, date, and author
\title{Exercise 6}
\author{Your Name, Collaborator's Name}
\date{\today}

\titleformat{\section}
  {\normalfont\normalsize\bfseries} % Format: font style, size, and weight
  {\thesection}{1em} % Label format and spacing
  {}
  \renewcommand{\thesubsection}{\thesection.\alph{subsection}}

\titleformat{\subsection}
  {\normalfont\small\bfseries} % Format: font style, size, and weight
  {\thesubsection}{1em} % Label format and spacing
  {}
\titleformat{\subsubsection}
  {\normalfont\small\bfseries} % Format: font style, size, and weight
  {\thesubsubsection}{1em} % Label format and spacing
  {}

\begin{document}
\begin{titlepage}
    \centering
    \vspace*{1in}
    
    {\Huge\bfseries Exercise 6\par}
    \vspace{1.5cm}
    {\Large \today\par}
    \vspace{1.5cm}
    {\Large\itshape Antonio Pampalone 23586519 \\ Giuseppe Pisante 23610012\\ Martina Raffaelli 23616907 \par}
    
    \vfill
    \includegraphics[width=0.3\textwidth]{FAU-Logo.png}\par\vspace{1cm} % Adjust the width as needed
   
\end{titlepage}

\newpage
\small

\section{Finite-difference method on a non-uniform grip}

\subsection{Transformation from from physical space into computational space}
Using the chain rule, the first derivative of \(\Phi\) with respect to \(x\) is given by:

\[
\frac{d\Phi}{dx} = \frac{d\Phi}{d\xi} \cdot \frac{d\xi}{dx}
\]
Since we assume that \(\xi\) is the computational space where grid points are equally spaced, we express \( d\xi/dx \) in terms of \( x \):

\[
\frac{d\xi}{dx} = \frac{1}{\frac{dx}{d\xi}}
\]
Substituting this into the previous equation:

\[
\frac{d\Phi}{dx} = \frac{d\Phi}{d\xi} \cdot \frac{1}{\frac{dx}{d\xi}}
\]
Thus, we obtain the compact form:

\begin{equation}
\frac{d\Phi}{dx} = \frac{\frac{d\Phi}{d\xi}}{\frac{dx}{d\xi}}.
\end{equation}
The second derivative is defined as:

\[
\frac{d^2\Phi}{dx^2} = \frac{d}{dx} \left( \frac{d\Phi}{dx} \right).
\]
Substituting \(\frac{d\Phi}{dx} = \frac{\frac{d\Phi}{d\xi}}{\frac{dx}{d\xi}}\):

\[
\frac{d^2\Phi}{dx^2} = \frac{d}{dx} \left( \frac{\frac{d\Phi}{d\xi}}{\frac{dx}{d\xi}} \right).
\]
Using the chain rule:

\[
\frac{d^2\Phi}{dx^2} = \frac{d}{d\xi} \left( \frac{\frac{d\Phi}{d\xi}}{\frac{dx}{d\xi}} \right) \cdot \frac{d\xi}{dx}.
\]
Again, substituting \( \frac{d\xi}{dx} = \frac{1}{\frac{dx}{d\xi}} \), we get:

\[
\frac{d^2\Phi}{dx^2} = \frac{\frac{d}{d\xi} \left( \frac{\frac{d\Phi}{d\xi}}{\frac{dx}{d\xi}} \right)}{\frac{dx}{d\xi}}.
\]
Now, applying the quotient rule to differentiate:

\[
\frac{d}{d\xi} \left( \frac{\frac{d\Phi}{d\xi}}{\frac{dx}{d\xi}} \right) =
\frac{\frac{d^2\Phi}{d\xi^2} \cdot \frac{dx}{d\xi} - \frac{d\Phi}{d\xi} \cdot \frac{d^2x}{d\xi^2}}{\left( \frac{dx}{d\xi} \right)^2}.
\]
Substituting this result:

\[
\frac{d^2\Phi}{dx^2} = \frac{\frac{\frac{d^2\Phi}{d\xi^2} \cdot \frac{dx}{d\xi} - \frac{d\Phi}{d\xi} \cdot \frac{d^2x}{d\xi^2}}{\left( \frac{dx}{d\xi} \right)^2}}{\frac{dx}{d\xi}}.
\]
Multiplying by \( \frac{1}{\frac{dx}{d\xi}} \), we obtain:

\begin{equation}
\frac{d^2\Phi}{dx^2} = \frac{\frac{d^2\Phi}{d\xi^2}}{\left( \frac{dx}{d\xi} \right)^2} - \frac{\frac{d^2x}{d\xi^2} \cdot \frac{d\Phi}{dx}}{\left( \frac{dx}{d\xi} \right)^2}.
\end{equation}
These expressions can be used to transform differential equations from physical space into computational space.

\subsection{Non-uniform grid}
Non-uniform grids are often used because they allow for adaptive resolution in areas where higher accuracy or detail is needed 
such as boundary layers, shocks, or vortices without increasing computational cost significantly across the entire domain. 
By refining the grid only where necessary, non-uniform grids reduce the number of total grid points, saving memory and computational time.
Furthermore non-uniform grids are better suited for domains with irregular or complex geometries since the grid can better conform to 
the shape of the domain, improving accuracy in boundary condition enforcement.

\subsection{Central finite-difference approximation on an non-equispaced grid}
To derive central finite-difference approximations for the first and second derivatives in the computational space \(\xi\) with second-order 
accuracy, we start from equation (1) and (2). For the first derivative we start computing the central finite-difference approximation for the term:

\[
\frac{d\Phi}{d\xi} \approx \frac{\Phi_{i+1} - \Phi_{i-1}}{2\Delta \xi},
\]
where \(\Delta \xi = 1\). Here \(\frac{dx}{d\xi}\) is computed based on the grid points \(x_i\).

\[
\frac{dx}{d\xi} \approx \frac{x_{i+1} - x_{i-1}}{2\Delta \xi}.
\]
Substituting into the equation (1) we get:

\begin{equation}
\frac{d\Phi}{dx} \approx \frac{\frac{\Phi_{i+1} - \Phi_{i-1}}{2}}{\frac{x_{i+1} - x_{i-1}}{2}} = \frac{\Phi_{i+1} - \Phi_{i-1}}{x_{i+1} - x_{i-1}}.
\end{equation}
Similary, to derive an approximation for the second derivative, we start computing the central finite-difference approximation for the term:

\[
\frac{d^2\Phi}{d\xi^2} \approx \frac{\Phi_{i+1} - 2\Phi_i + \Phi_{i-1}}{\Delta \xi^2}.
\]
where \(\Delta \xi = 1\). Here \(\frac{d^2x}{d\xi^2}\) is computed based on the grid points \(x_i\).

\[
\frac{d^2x}{d\xi^2} \approx x_{i+1} - 2x_i + x_{i-1}.
\]
Substituting into the equation (2) we get:

\begin{equation}
\frac{d^2\Phi}{dx^2} \approx \frac{\Phi_{i+1} - 2\Phi_i + \Phi_{i-1}}{\left(\frac{x_{i+1} - x_{i-1}}{2}\right)^2} - \frac{\left(x_{i+1} - 2x_i + x_{i-1}\right) \cdot \frac{\Phi_{i+1} - \Phi_{i-1}}{x_{i+1} - x_{i-1}}}{\left(\frac{x_{i+1} - x_{i-1}}{2}\right)^2}.
\end{equation}

\subsection{Discretized steady one-dimensional advection-diffusion equation}
Using central finite-difference approximations for the first and second derivatives on a non-uniform grid, we want to discretize 
the steady one-dimensional advection-diffusion equation. Substituting equation (3) and (4) into the advection-diffusion equation leads to:
\[
\frac{\Phi_{i+1} - \Phi_{i-1}}{x_{i+1} - x_{i-1}} = \frac{1}{\text{Pe}} \left[ \frac{\Phi_{i+1} - 2\Phi_i + \Phi_{i-1}}{\left(\frac{x_{i+1} - x_{i-1}}{2}\right)^2} - \frac{\left(x_{i+1} - 2x_i + x_{i-1}\right) \cdot \frac{\Phi_{i+1} - \Phi_{i-1}}{x_{i+1} - x_{i-1}}}{\left(\frac{x_{i+1} - x_{i-1}}{2}\right)^2} \right].
\]
Rearranging the above equation, the discretized equation for an interior point \(i\) becomes:

\[
a_{i-1} \Phi_{i-1} + b_i \Phi_i + c_{i+1} \Phi_{i+1} = 0,
\]
where:
\[
a_{i-1} = \frac{-1}{\Delta x_i} + \frac{1}{\text{Pe}} \left( \frac{1}{\Delta x_i^2} - \frac{x_{i+1} - 2x_i + x_{i-1}}{\Delta x_i^2} \right),
\]
\[
b_i = \frac{2}{\text{Pe} \cdot \Delta x_i^2},
\]
\[
c_{i+1} = \frac{1}{\Delta x_i} + \frac{1}{\text{Pe}} \left( \frac{1}{\Delta x_i^2} + \frac{x_{i+1} - 2x_i + x_{i-1}}{\Delta x_i^2} \right).
\]
Here, \(\Delta x_i = x_{i+1} - x_{i-1}\).
The discretized system can be expressed in matrix form as:
\[
\mathbf{A} \boldsymbol{\Phi} = \mathbf{b},
\]
where:
\begin{itemize}
    \item \(\mathbf{A}\) is an \(N \times N\) sparse coefficient matrix,
    \item \(\boldsymbol{\Phi} = [\Phi_1, \Phi_2, \dots, \Phi_N]^T\) is the vector of unknowns,
    \item \(\mathbf{b}\) is the source term (zero for homogeneous equations).
\end{itemize}
For \(N\) grid points, the coefficient matrix \(\mathbf{A}\) is tridiagonal for interior points:

\[
\mathbf{A} =
\begin{bmatrix}
b_1 & c_2 & 0 & \cdots & 0 \\
a_2 & b_2 & c_3 & \cdots & 0 \\
0 & a_3 & b_3 & \cdots & 0 \\
\vdots & \vdots & \vdots & \ddots & c_{N-1} \\
0 & 0 & 0 & a_{N} & b_{N}
\end{bmatrix}.
\]







\begin{thebibliography}{9}
    \bibitem{GitHubRepo}
    \textit{CFD Repository},\\
    Available at: \url{https://github.com/GiuseppePisante/CFD.git}
    
    \bibitem{GitHubCopilot}
    \textit{GitHub Copilot},\\
    GitHub. Available at: \url{https://github.com/features/copilot}
  \end{thebibliography}
  
  \end{document}