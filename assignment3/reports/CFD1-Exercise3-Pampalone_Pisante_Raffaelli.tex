\documentclass{article}
\usepackage{amsmath}
\usepackage{titlesec}
\usepackage{graphicx}
\usepackage[margin=1in]{geometry}
\usepackage{hyperref}

% Title, date, and author
\title{Exercise 3}
\author{Your Name, Collaborator's Name}
\date{\today}

\titleformat{\section}
  {\normalfont\normalsize\bfseries} % Format: font style, size, and weight
  {\thesection}{1em} % Label format and spacing
  {}
  \renewcommand{\thesubsection}{\thesection.\alph{subsection}}

\titleformat{\subsection}
  {\normalfont\small\bfseries} % Format: font style, size, and weight
  {\thesubsection}{1em} % Label format and spacing
  {}
\titleformat{\subsubsection}
  {\normalfont\small\bfseries} % Format: font style, size, and weight
  {\thesubsubsection}{1em} % Label format and spacing
  {}

\begin{document}
\begin{titlepage}
    \centering
    \vspace*{1in}
    
    {\Huge\bfseries Exercise 3\par}
    \vspace{1.5cm}
    {\Large \today\par}
    \vspace{1.5cm}
    {\Large\itshape Antonio Pampalone 23586519 \\ Giuseppe Pisante 23610012\\ Martina Raffaelli 23616907 \par}
    
    \vfill
    \includegraphics[width=0.3\textwidth]{FAU-Logo.png}\par\vspace{1cm} % Adjust the width as needed
   
\end{titlepage}

\newpage
\small
\section{Finite difference approximation of the second derivative}
\subsection{fourth-order central finite-difference approximation}
We derive the fourth-order central finite-difference approximation for \( \frac{d^2f}{dx^2} \) using values of \( f \) at \( x_i \), \( x_{i-1} \), \( x_{i+1} \), \( x_{i-2} \), and \( x_{i+2} \), assuming a uniform grid spacing \( \Delta x \), as provided by the text.

We then expand \( f(x) \) at the neighboring points around \( x_i \) using Taylor series:

\begin{align*}
f(x_{i+1}) &= f(x_i) + f'(x_i)\Delta x + \frac{f''(x_i)}{2!}(\Delta x)^2 + \frac{f'''(x_i)}{3!}(\Delta x)^3 + \frac{f^{(4)}(x_i)}{4!}(\Delta x)^4 + \dots, \\
f(x_{i+2}) &= f(x_i) + 2f'(x_i)\Delta x + \frac{4f''(x_i)}{2!}(\Delta x)^2 + \frac{8f'''(x_i)}{3!}(\Delta x)^3 + \frac{16f^{(4)}(x_i)}{4!}(\Delta x)^4 + \dots, \\
f(x_{i-1}) &= f(x_i) - f'(x_i)\Delta x + \frac{f''(x_i)}{2!}(\Delta x)^2 - \frac{f'''(x_i)}{3!}(\Delta x)^3 + \frac{f^{(4)}(x_i)}{4!}(\Delta x)^4 - \dots, \\
f(x_{i-2}) &= f(x_i) - 2f'(x_i)\Delta x + \frac{4f''(x_i)}{2!}(\Delta x)^2 - \frac{8f'''(x_i)}{3!}(\Delta x)^3 + \frac{16f^{(4)}(x_i)}{4!}(\Delta x)^4 - \dots.
\end{align*}


As a further step, we add the symmetric spatial terms to cancel the odd derivatives:

\begin{align*}
f(x_{i+1}) + f(x_{i-1}) &= 2f(x_i) + \frac{2f''(x_i)}{2!}(\Delta x)^2 + \frac{2f^{(4)}(x_i)}{4!}(\Delta x)^4 + \dots, \\
f(x_{i+2}) + f(x_{i-2}) &= 2f(x_i) + \frac{8f''(x_i)}{2!}(\Delta x)^2 + \frac{32f^{(4)}(x_i)}{4!}(\Delta x)^4 + \dots.
\end{align*}


To approximate \( \frac{d^2f}{dx^2} \), we form a weighted sum of these terms:

\[
\frac{d^2f}{dx^2} = a [f(x_{i+1}) + f(x_{i-1})] + b [f(x_{i+2}) + f(x_{i-2})] + c f(x_i),
\]

where \( a, b, c \) are coefficients to be determined. \\
We substitute the Taylor expansions into \( S \):

\begin{align*}
  \frac{d^2f}{dx^2} &= 2a f(x_i) + 2b f(x_i) + c f(x_i) \\
  &\quad + \left( \frac{2a}{2!} (\Delta x)^2 + \frac{8b}{2!} (\Delta x)^2 \right) f''(x_i) \\
  &\quad + \left( \frac{2a}{4!} (\Delta x)^4 + \frac{32b}{4!} (\Delta x)^4 \right) f^{(4)}(x_i) + \dots
\end{align*}

Match coefficients for \( f(x_i), f''(x_i), \) and higher-order terms to ensure:
The coefficients are determined by matching terms in the Taylor series expansions:

\begin{enumerate}
  \item Coefficient of \( f(x_i) \):
  \[
  2a + 2b + c = 0
  \]
  \item Coefficient of \( f''(x_i) \): This should equal \( \frac{1}{(\Delta x)^2} \), so:
  \[
  a \cdot \frac{2(\Delta x)^2}{2!} + b \cdot \frac{8(\Delta x)^2}{2!} = \frac{1}{(\Delta x)^2}
  \]
  Simplifying:
  \[
  a + 4b = 6
  \]
  \item Coefficient of \( f^{(4)}(x_i) \): This term should be eliminated to ensure fourth-order accuracy:
  \[
  a \cdot \frac{2(\Delta x)^4}{4!} + b \cdot \frac{32(\Delta x)^4}{4!} = 0
  \]
  Simplifying:
  \[
  a + 16b = 0
  \]
\end{enumerate}


After solving the system of equations, the coefficients are:

\[
a = 16, \quad b = -1, \quad c = -30.
\]

We substitute these coefficients to get the fourth-order finite-difference approximation for \( \frac{d^2f}{dx^2} \):

\[
\frac{d^2f}{dx^2} \approx \frac{-f(x_{i+2}) + 16f(x_{i+1}) - 30f(x_i) + 16f(x_{i-1}) - f(x_{i-2})}{12 (\Delta x)^2}.
\]


\subsection{Second-order one-sided finite-difference approximation for \texorpdfstring{$\frac{\partial^2 u}{\partial x^2}$}{d2u/dx2}}
The backward second-order one-sided finite difference approximation for the second derivative \( \frac{d^2f}{dx^2} \) is derived as follows.
Expand \( f(x_{i-1}) \) and \( f(x_{i-2}) \) around \( x_i \) using Taylor series:
\[
f(x_{i-1}) = f(x_i) - f'(x_i) \Delta x + \frac{f''(x_i)}{2} (\Delta x)^2 - \frac{f^{(3)}(x_i)}{6} (\Delta x)^3 + O((\Delta x)^4),
\]
\[
f(x_{i-2}) = f(x_i) - 2f'(x_i) \Delta x + 2 \cdot \frac{f''(x_i)}{2} (\Delta x)^2 - \frac{2 \cdot f^{(3)}(x_i)}{6} (\Delta x)^3 + O((\Delta x)^4).
\]

We then subtract the first expansion from the second expansion:
\[
f(x_{i-2}) - 2f(x_{i-1}) + f(x_i) = \left( f(x_i) - 2f'(x_i) \Delta x + \frac{2 f''(x_i)}{2} (\Delta x)^2 + \dots \right)
- 2\left( f(x_i) - f'(x_i) \Delta x + \frac{f''(x_i)}{2} (\Delta x)^2 + \dots \right)
+ f(x_i).
\]

Simplifying:
\[
f(x_{i-2}) - 2f(x_{i-1}) + f(x_i) = \frac{f''(x_i)}{2} (\Delta x)^2 + O((\Delta x)^3).
\]

Thus, the backward second-order one-sided finite difference approximation for the second derivative is:
\[
\frac{d^2f}{dx^2} \approx \frac{f(x_i) - 2f(x_{i-1}) + f(x_{i-2})}{(\Delta x)^2}.
\]

This method is second-order accurate, with the leading error term proportional to \( O((\Delta x)^2) \).
It is important to note that this process can be applied to the forward difference scheme as well, with the same accuracy. In this case, the second derivative would be approximated as:
\[
\frac{d^2f}{dx^2} \approx \frac{f(x_{i+2}) - 2f(x_{i+1}) + f(x_i)}{(\Delta x)^2}.
\]


\subsection{Situations in which we need one-sided approximations}
One-sided approximations are needed when the function is not defined on both sides of the point where the derivative is being calculated.

\section{Finite difference approximation of the third and mixed derivatives}

\subsection{Second-order finite-difference approximation for \texorpdfstring{$\frac{\partial^3 \Psi}{\partial y^3}$}{d3Psi/dy3}}
For the third derivative in \(y\), apply the central difference formula iteratively. Combining terms gives:
\[
\frac{\partial^3 \Psi}{\partial y^3} \approx \frac{\Psi_{i+2, j} - 2\Psi_{i+1, j} + 2\Psi_{i-1, j} - \Psi_{i-2, j}}{2 \Delta y^3}.
\]

\subsection{Second-order finite-difference approximation for \texorpdfstring{$\frac{\partial^2 \Psi}{\partial y \partial x}$}{d2Psi/dy dx}}
The mixed derivative is computed using central differences in both \(x\) and \(y\). First, compute the difference along \(x\):
\[
\frac{\Psi_{i+1, j} - \Psi_{i-1, j}}{2\Delta x}.
\]
Then compute the difference along \(y\):
\[
\frac{\partial^2 \Psi}{\partial y \partial x} \approx \frac{\Psi_{i+1, j+1} - \Psi_{i-1, j+1} - \Psi_{i+1, j-1} + \Psi_{i-1, j-1}}{4 \Delta x \Delta y}.
\]




\section{Steady temperature distribution of an alluminium bar}

\subsection{Non-dimensional form of the heat equation}

The time-dependent heat equation in 1D is:

\[
\rho c_p \frac{\partial T}{\partial t} = \lambda \frac{\partial^2 T}{\partial x^2}
\]

In this formulation, we have replaced the temperature \(T\) with the function \(u(x, t)\).

\[
\frac{\partial u}{\partial t} = \alpha \frac{\partial^2 u}{\partial x^2}
\]

where \(\alpha\) is the thermal diffusivity defined as:

\[
\alpha = \frac{\lambda}{c \rho} 
\]

To derive the non-dimensional form of the heat equation we have to introduce the dimensionless variables \(\hat{x}\), \(\hat{t}\),
and \(\hat{u}\), respectively, for position, time and temperature:

\[
\hat{x} = \frac{x}{L^*}, \quad \hat{t} = \frac{t}{T^*}, \quad \hat{u}(\hat{x}, \hat{t}) = \frac{u(x, t)}{U^*}
\]

Since the characteristic length, time, and temperature are \( L^* \), \( T^* \), and \( U^* \), respectively, with dimensions:

\[
[L^*] = L, \quad [T^*] = T, \quad [U^*] = U.
\]

The choice for the characteristic length is \( L^* = l \), the length of the rod, in this way, while \( x \) is in the range 
\( 0 < x < l \), \(\hat{x}\) is in the range \( 0 < \hat{x} < 1 \). In the following passages taken from the source \cite{HeatEquation}, we have:

\[
\frac{\partial u}{\partial t} = \frac{\partial \hat{u}}{\partial \hat{t}} \cdot \frac{U^*}{T^*} = \frac{U^*}{T^*} \frac{\partial \hat{u}}{\partial \hat{t}},
\]
\[
\frac{\partial u}{\partial x} = \frac{\partial \hat{u}}{\partial \hat{x}} \cdot \frac{U^*}{L^*} = \frac{U^*}{L^*} \frac{\partial \hat{u}}{\partial \hat{x}},
\]
\[
\frac{\partial^2 u}{\partial x^2} = \frac{U^*}{L^*} \cdot \frac{1}{L^*} \frac{\partial^2 \hat{u}}{\partial \hat{x}^2} = \frac{U^*}{L^2} \frac{\partial^2 \hat{u}}{\partial \hat{x}^2}.
\]

Substituting these into the heat equation gives:
\[
\frac{U^*}{T^*} \frac{\partial \hat{u}}{\partial \hat{t}} = \alpha \frac{U^*}{L^2} \frac{\partial^2 \hat{u}}{\partial \hat{x}^2}.
\]

To make the PDE simpler, we choose for the characteristic time \( T^* = \frac{L^2}{\alpha}\), so that the equation becomes:
\[
\frac{\partial \hat{u}}{\partial \hat{t}} = \frac{\partial^2 \hat{u}}{\partial \hat{x}^2}
\]

The most important non-dimensional number for this problem is the Fourier number wich characterizes 
the ratio of conductive heat transfer to thermal storage and is used to analyze heat conduction in this context.

\[
Fo = \frac{\lambda t}{\rho c_p L^2}
\]

\subsection{Dimensional form of the equation for the numerical solution}

For our numerical solution we want to use the dimensional form of the equation because in this way, we can maintain physical units while 
controlling stability numerically. For example, the Explicit Euler method for time-stepping and central difference scheme for space gives:
\[
\rho c_p \frac{T_i^{n+1} - T_i^n}{\Delta t} = \lambda \frac{T_{i+1}^n - 2T_i^n + T_{i-1}^n}{\Delta x^2} + q
\]

The time step $\Delta t$ and spatial interval $\Delta x$ must satisfy the Von Neumann stability criterion:
\[
\Delta t \leq \frac{\rho c_p (\Delta x)^2}{2 \lambda}
\]

In this way, it is possible to define a grid in terms of the effective length $L$ and the time interval $\Delta t$, 
thereby maintaining physical units. The values of $\rho$, $c_p$, $\lambda$, and $q$ can be directly substituted into 
the discretized equation in their dimensional form, and the boundary conditions can be applied directly. Furthermore,
once the numerical solution is computed, the temperature at all nodes $T_i^n$ is obtained directly in Kelvin, preserving
dimensional consistency. While non-dimensionalization may simplify some analyses, it is not strictly necessary for straightforward 
numerical implementation.

\subsection{Discretization using a second-order central differende scheme in space}

At boundaries, it is not possible to use the central difference formula directly because one or both adjacent points fall outside the domain.
The loop we implemented on python skips the first and last point because the second derivative cannot be computed at the boundaries using CDS
without additional information. Boundary conditions can be added separately after calculating the second derivative for the internal elements.


\begin{thebibliography}{9}
  \bibitem{GitHubRepo}
  \textit{CFD Repository},\\
  Available at: \url{https://github.com/GiuseppePisante/CFD.git}
  
  \bibitem{GitHubCopilot}
  \textit{GitHub Copilot},\\
  GitHub. Available at: \url{https://github.com/features/copilot}
  
  \bibitem{HeatEquation}
  MIT OpenCourseWare,\\
  \textit{Heat Equation Notes}, 2006,\\
  Available at: \url{https://ocw.mit.edu/courses/18-303-linear-partial-differential-equations-fall-2006/d11b374a85c3fde55ec971fe587f8a50_heateqni.pdf},\\
\end{thebibliography}

\end{document}