\documentclass{article}
\usepackage{amsmath}
\usepackage{titlesec}
\usepackage{graphicx}
\usepackage[margin=1in]{geometry}
\usepackage{hyperref}

% Title, date, and author
\title{Exercise 3}
\author{Your Name, Collaborator's Name}
\date{\today}

\titleformat{\section}
  {\normalfont\normalsize\bfseries} % Format: font style, size, and weight
  {\thesection}{1em} % Label format and spacing
  {}
  \renewcommand{\thesubsection}{\thesection.\alph{subsection}}

\titleformat{\subsection}
  {\normalfont\small\bfseries} % Format: font style, size, and weight
  {\thesubsection}{1em} % Label format and spacing
  {}
\titleformat{\subsubsection}
  {\normalfont\small\bfseries} % Format: font style, size, and weight
  {\thesubsubsection}{1em} % Label format and spacing
  {}

\begin{document}
\begin{titlepage}
    \centering
    \vspace*{1in}
    
    {\Huge\bfseries Exercise 3\par}
    \vspace{1.5cm}
    {\Large \today\par}
    \vspace{1.5cm}
    {\Large\itshape Antonio Pampalone 23586519 \\ Giuseppe Pisante 23610012\\ Martina Raffaelli 23616907 \par}
    
    \vfill
    \includegraphics[width=0.3\textwidth]{FAU-Logo.png}\par\vspace{1cm} % Adjust the width as needed
   
\end{titlepage}

\newpage
\small

\section{Steady temperature distribution of an alluminium bar}

\subsection{Non-dimensional form of the heat equation}

The time-dependent heat equation in 1D is:

\[
\rho c_p \frac{\partial T}{\partial t} = \lambda \frac{\partial^2 T}{\partial x^2}
\]

In this formulation, we have replaced the temperature \(T\) with the function \(u(x, t)\).

\[
\frac{\partial u}{\partial t} = \alpha \frac{\partial^2 u}{\partial x^2}
\]

where \(\alpha\) is the thermal diffusivity defined as:

\[
\alpha = \frac{\lambda}{c \rho} 
\]

To derive the non-dimensional form of the heat equation we have to introduce the dimensionless variables \(\hat{x}\), \(\hat{t}\),
and \(\hat{u}\), respectively, for position, time and temperature:

\[
\hat{x} = \frac{x}{L^*}, \quad \hat{t} = \frac{t}{T^*}, \quad \hat{u}(\hat{x}, \hat{t}) = \frac{u(x, t)}{U^*}
\]

Since the characteristic length, time, and temperature are \( L^* \), \( T^* \), and \( U^* \), respectively, with dimensions:

\[
[L^*] = L, \quad [T^*] = T, \quad [U^*] = U.
\]

The choice for the characteristic length is \( L^* = l \), the length of the rod, in this way, while \( x \) is in the range 
\( 0 < x < l \), \(\hat{x}\) is in the range \( 0 < \hat{x} < 1 \). In the following passages taken from the source \cite{HeatEquation}, we have:

\[
\frac{\partial u}{\partial t} = \frac{\partial \hat{u}}{\partial \hat{t}} \cdot \frac{U^*}{T^*} = \frac{U^*}{T^*} \frac{\partial \hat{u}}{\partial \hat{t}},
\]
\[
\frac{\partial u}{\partial x} = \frac{\partial \hat{u}}{\partial \hat{x}} \cdot \frac{U^*}{L^*} = \frac{U^*}{L^*} \frac{\partial \hat{u}}{\partial \hat{x}},
\]
\[
\frac{\partial^2 u}{\partial x^2} = \frac{U^*}{L^*} \cdot \frac{1}{L^*} \frac{\partial^2 \hat{u}}{\partial \hat{x}^2} = \frac{U^*}{L^2} \frac{\partial^2 \hat{u}}{\partial \hat{x}^2}.
\]

Substituting these into the heat equation gives:
\[
\frac{U^*}{T^*} \frac{\partial \hat{u}}{\partial \hat{t}} = \alpha \frac{U^*}{L^2} \frac{\partial^2 \hat{u}}{\partial \hat{x}^2}.
\]

To make the PDE simpler, we choose for the characteristic time \( T^* = \frac{L^2}{\alpha}\), so that the equation becomes:
\[
\frac{\partial \hat{u}}{\partial \hat{t}} = \frac{\partial^2 \hat{u}}{\partial \hat{x}^2}
\]

The most important non-dimensional number for this problem is the Fourier number wich characterizes 
the ratio of conductive heat transfer to thermal storage and is used to analyze heat conduction in this context.

\[
Fo = \frac{\lambda t}{\rho c_p L^2}
\]

\subsection{Dimensional form of the equation for the numerical solution}

For our numerical solution we want to use the dimensional form of the equation because in this way, we can maintain physical units while 
controlling stability numerically. For example, the Explicit Euler method for time-stepping and central difference scheme for space gives:
\[
\rho c_p \frac{T_i^{n+1} - T_i^n}{\Delta t} = \lambda \frac{T_{i+1}^n - 2T_i^n + T_{i-1}^n}{\Delta x^2} + q
\]

The time step $\Delta t$ and spatial interval $\Delta x$ must satisfy the Von Neumann stability criterion:
\[
\Delta t \leq \frac{\rho c_p (\Delta x)^2}{2 \lambda}
\]

In this way, it is possible to define a grid in terms of the effective length $L$ and the time interval $\Delta t$, 
thereby maintaining physical units. The values of $\rho$, $c_p$, $\lambda$, and $q$ can be directly substituted into 
the discretized equation in their dimensional form, and the boundary conditions can be applied directly. Furthermore,
once the numerical solution is computed, the temperature at all nodes $T_i^n$ is obtained directly in Kelvin, preserving
dimensional consistency. While non-dimensionalization may simplify some analyses, it is not strictly necessary for straightforward 
numerical implementation.

\subsection{Discretization using a second-order central differende scheme in space}

At boundaries, it is not possible to use the central difference formula directly because one or both adjacent points fall outside the domain.
The loop we implemented on python skips the first and last point because the second derivative cannot be computed at the boundaries using CDS
without additional information. Boundary conditions can be added separately after calculating the second derivative for the internal elements.


\begin{thebibliography}{9}
  \bibitem{GitHubRepo}
  \textit{CFD Repository},\\
  Available at: \url{https://github.com/GiuseppePisante/CFD.git}
  
  \bibitem{GitHubCopilot}
  \textit{GitHub Copilot},\\
  GitHub. Available at: \url{https://github.com/features/copilot}
  
  \bibitem{HeatEquation}
  MIT OpenCourseWare,\\
  \textit{Heat Equation Notes}, 2006,\\
  Available at: \url{https://ocw.mit.edu/courses/18-303-linear-partial-differential-equations-fall-2006/d11b374a85c3fde55ec971fe587f8a50_heateqni.pdf},\\
\end{thebibliography}

\end{document}