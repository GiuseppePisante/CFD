\documentclass{article}
\usepackage{amsmath}
\usepackage{titlesec}
\usepackage{graphicx}
\usepackage[margin=1in]{geometry}
\usepackage{hyperref}

% Title, date, and author
\title{Exercise 2}
\author{Your Name, Collaborator's Name}
\date{\today}

\titleformat{\section}
  {\normalfont\normalsize\bfseries} % Format: font style, size, and weight
  {\thesection}{1em} % Label format and spacing
  {}
  \renewcommand{\thesubsection}{\thesection.\alph{subsection}}

\titleformat{\subsection}
  {\normalfont\small\bfseries} % Format: font style, size, and weight
  {\thesubsection}{1em} % Label format and spacing
  {}
\titleformat{\subsubsection}
  {\normalfont\small\bfseries} % Format: font style, size, and weight
  {\thesubsubsection}{1em} % Label format and spacing
  {}

\begin{document}
\begin{titlepage}
    \centering
    \vspace*{1in}
    
    {\Huge\bfseries Exercise 1\par}
    \vspace{1.5cm}
    {\Large \today\par}
    \vspace{1.5cm}
    {\Large\itshape Antonio Pampalone 23586519 \\ Giuseppe Pisante 23610012\\ Martina Raffaelli 23616907 \par}
    
    \vfill
    \includegraphics[width=0.3\textwidth]{FAU-Logo.png}\par\vspace{1cm} % Adjust the width as needed
   
\end{titlepage}

\newpage
\small


\section{Fundamentals of Differential Equations}

\subsection{Difference between ordinary derivative, partial derivative, and material (total) derivative}
\begin{itemize}
    \item \textbf{Ordinary derivative} \( \left( \frac{d}{dt} \right) \): Describes the rate of change of a function with respect to one variable. It is used for functions depending on a single variable, such as \( f(t) \).
    \item \textbf{Partial derivative} \( \left( \frac{\partial}{\partial t} \right) \): Describes the rate of change of a multivariable function with respect to one of its variables, while holding other variables constant. This is often used in multivariable functions such as \( f(x, t) \), where we can find \( \frac{\partial f}{\partial t} \) while \( x \) remains fixed.
    \item \textbf{Material (total) derivative} \( \left( \frac{D}{Dt} \right) \): is a measure of the rate of change of a physical quantity (like velocity or temperature) experienced by an observer moving with the fluid. It combines both local and convective rates of change as, for example, in a function \( f(x, t) \), \( \frac{D}{Dt} = \frac{\partial}{\partial t} + u \frac{\partial}{\partial x} \) for some velocity field \( u \).
\end{itemize}

\subsection{Ordinary and partial differential equations}
\begin{itemize}
    \item \textbf{Ordinary Differential Equations (ODEs)}: These involve derivatives with respect to a single variable. For example, \( \frac{dy}{dt} = y \) is an ODE.
    \item \textbf{Partial Differential Equations (PDEs)}: These involve partial derivatives with respect to multiple variables. For instance, the heat equation \( \frac{\partial u}{\partial t} = \alpha \frac{\partial^2 u}{\partial x^2} \) is a PDE.
\end{itemize}

\subsection{Order of a differential equation}
The order of a differential equation is the highest order of derivative present in the equation.
\begin{itemize}
    \item \textbf{First-order ODE}: \( \frac{dy}{dt} = ky \).
    \item \textbf{Second-order PDE}: The wave equation \( \frac{\partial^2 u}{\partial t^2} = c^2 \frac{\partial^2 u}{\partial x^2} \).
    \item \textbf{Third-order ODE}: \( \frac{d^3 y}{dt^3} + \frac{d^2 y}{dt^2} + y = 0 \).
\end{itemize}

\subsection{Linear and non-linear differential equations}
\begin{itemize}
    \item \textbf{Linear Differential Equations}: These have terms that are linear in the unknown function and its derivatives. For example, \( \frac{dy}{dt} + 3y = 0 \) is linear.
    \item \textbf{Non-linear Differential Equations}: These have terms that are non-linear in the unknown function or its derivatives. For instance, the Navier-Stokes equation \( \rho \left( \frac{\partial \mathbf{u}}{\partial t} + \mathbf{u} \cdot \nabla \mathbf{u} \right) = -\nabla p + \mu \nabla^2 \mathbf{u} + \mathbf{f} \) is non-linear. This non-linearity arises due to the convective term \( \mathbf{u} \cdot \nabla \mathbf{u} \), which represents the interaction of the velocity field with itself. Specifically, \( \mathbf{u} \cdot \nabla \mathbf{u} \) is non-linear because it involves the product of the velocity field \( \mathbf{u} \) with its own gradient \( \nabla \mathbf{u} \).
\end{itemize}

\subsection{Initial value problem (IVP) and boundary value problem (BVP)}
\begin{itemize}
    \item \textbf{Initial Value Problem (IVP)}: A problem that requires solving a differential equation with specified initial conditions, such as \( y(0) = y_0 \), in time.
    \item \textbf{Boundary Value Problem (BVP)}: A problem where the solution to a differential equation is sought within a specified range, with conditions, usually Dirichlet or Neumann, given at the boundaries of the range, like \( u(0) = 0 \) and \( u(1) = 1 \).
\end{itemize}

\subsection{Parabolic and elliptic PDE examples and their conditions}
The difference between parabolic and elliptic PDEs can be defined through the computation of a discriminant \( \Delta = b^2 - 4ac \), where \( a \), \( b \), and \( c \) are coefficients from the second-order PDE of the form \( a \frac{\partial^2 u}{\partial x^2} + b \frac{\partial^2 u}{\partial x \partial y} + c \frac{\partial^2 u}{\partial y^2} + \ldots = 0 \). If \( \Delta = 0 \), the PDE is parabolic, and if \( \Delta < 0 \), the PDE is elliptic.
\begin{itemize}
    \item \textbf{Parabolic PDE}: The heat equation \( \frac{\partial u}{\partial t} = \alpha \frac{\partial^2 u}{\partial x^2} \) is parabolic and typically requires both initial and boundary conditions.
    \item \textbf{Elliptic PDE}: Laplace's equation \( \nabla^2 u = 0 \) is elliptic and usually requires boundary conditions but not initial conditions, since it does not depend on time.
\end{itemize}

\section{Order Reduction}

The governing equation of the damped oscillator is given by:
\begin{equation}
m \frac{d^2 y(t)}{dt^2} + b \frac{dy(t)}{dt} + c y(t) = 0
\end{equation}

with initial conditions:
\[
y(0) = s_0, \quad \frac{dy(0)}{dt} = v_0.
\]

We aim to transform this second-order ODE into a system of first-order differential equations. To reduce a second-order ODE to a 
system of first-order ODEs we can introduce new variables to represent the derivatives of the function \( y(t) \). 
In particular, we define:
\[
y_1(t) = y(t)
\]

and introduce a new variable \( y_2(t) \) to represent the first derivative of \( y(t) \):
\[
y_2(t) = \frac{dy(t)}{dt}.
\]

Since \( \frac{dy_2(t)}{dt} = \frac{d^2 y(t)}{dt^2} \), we can substitute this into the original equation to obtain:
\begin{equation}
m \frac{dy_2(t)}{dt} + b y_2(t) + c y_1(t) = 0.
\end{equation}

In this way, we can express the problem as two coupled first-order differential equations:
\[
\begin{cases}
\frac{dy_1(t)}{dt} &= y_2(t), \\
\frac{dy_2(t)}{dt} &= -\frac{b}{m} y_2(t) - \frac{c}{m} y_1(t).
\end{cases}
\]

However, we also need to rewrite the initial conditions for \( y(t) \) and \( \frac{dy(t)}{dt} \):
\begin{itemize}
    \item \( y_1(0) = s_0 \),
    \item \( y_2(0) = v_0 \).
\end{itemize}

This approach allows us to solve the system using methods suited for first-order differential equations, enabling easier numerical or analytical analysis.






\section{Blasius Equation}

\subsection*{Part (a): Convert the Blasius Equation to a System of First-Order ODEs}

The Blasius equation is given by:

\begin{equation}
f''' + \frac{1}{2} f f'' = 0
\end{equation}

with \( f' = \frac{u}{U_\infty} \). Three boundary conditions are necessary to solve this equation:
\begin{itemize}
    \item \( \eta = 0 \): \( f' = f = 0 \) (no-slip condition)
    \item \( \eta \to \infty \): \( f' = 1 \) (free outer flow)
\end{itemize}

We aim to transform this third-order ODE into a system of first-order differential equations. To reduce a third-order ODE to a 
system of first-order ODEs we can introduce new variables to represent the derivatives of the function \( f(\eta) \). 
In particular, we define:
\[
y_1 = f, \quad y_2 = f' = \frac{df}{d\eta}, \quad y_3 = f'' = \frac{d^2 f}{d\eta^2}
\]

Then, the derivatives of these variables with respect to \(\eta\) are:
\[
\frac{dy_1}{d\eta} = y_2, \quad \frac{dy_2}{d\eta} = y_3
\]

Now, we can substitute this into the original Blasius equation to obtain:
\[
\frac{dy_3}{d\eta} = -\frac{1}{2} y_1 y_3
\]

In this way, we can express the problem as three coupled first-order differential equations:
\[
\begin{cases}
\frac{dy_1}{d\eta} = y_2 \\
\frac{dy_2}{d\eta} = y_3 \\
\frac{dy_3}{d\eta} = -\frac{1}{2} y_1 y_3
\end{cases}
\]

with boundary conditions:
\begin{itemize}
    \item At \( \eta = 0 \): \( y_1 = 0 \), \( y_2 = 0 \)
    \item As \( \eta \to \infty \): \( y_2 = 1 \)
\end{itemize}

\subsection*{Part (b): Providing an Initial Condition for \( f''(0)\)}

To solve this problem as an initial value problem, we need an initial value for \( f''(0) \). However, the boundary condition \( y_2(\infty) = 1 \) is 
specified at infinity, making it impractical to impose this condition directly at a finite point. To address this, we can use an iterative approach:

\begin{enumerate}
    \item Guess an initial value for \( f''(0) \).
    \item Integrate the system of equations from \( \eta = 0 \) to a sufficiently large value of \( \eta \)  where \( y_2(\eta) \) approaches a constant. 
    To solve this system numerically, we use the Runge-Kutta method of fourth order (RK4) that allows to approximate solutions to ordinary differential equations.
    
    For a step size \( h \), the RK4 method computes the next values \( y_{i+1} \) as follows:
    
    \[
    k_1 = h \cdot f(\eta_i, y_i)
    \]
    \[
    k_2 = h \cdot f\left(\eta_i + \frac{h}{2}, y_i + \frac{k_1}{2}\right)
    \]
    \[
    k_3 = h \cdot f\left(\eta_i + \frac{h}{2}, y_i + \frac{k_2}{2}\right)
    \]
    \[
    k_4 = h \cdot f(\eta_i + h, y_i + k_3)
    \]
    
    The next value of the solution is updated by:
    
    \[
    y_{i+1} = y_i + \frac{1}{6} \left(k_1 + 2k_2 + 2k_3 + k_4\right)
    \]

    \item Check if \( y_2(\eta) \) approaches 1 as \( \eta \to \infty \). If \( y_2(\eta) \) is not close to 1, adjust the initial guess for \( y_3(0) \) 
    iterate this process until the condition \( y_2(\infty) = 1 \) (or close to it) is satisfied within a desired tolerance.
\end{enumerate}

This iterative approach allows us to find an appropriate initial condition for \( y_3(0) = f''(0) \) that satisfies the boundary condition at infinity.





\section{ Solving Blasius Equation numerically}

\subsection*{Part (a): Euler and RK-4 methods}
The attached file contains the Python code named Blasius.py, which should be executed to obtain the numerical solution to the Blasius equation."

\subsection*{Part (b): Recommendations for solving the Blausius equations}
For solving the Blasius equation, I recommend using the Runge-Kutta 4 (RK4) method with a a step size of \( \Delta \eta = 0.05 \).
The RK4 method provides greater accuracy compared to the Euler method, especially for nonlinear differential equations like the Blasius equation. 
If computational resources are limited, a slightly larger step size, such as can still provide acceptable results.

\subsection*{Part (c): Recommendations for solving the Blausius equations}
The velocity profile of the Blasius boundary layer consists of two main components: the streamwise velocity \( u^* = f'(\eta) \) and the wall-normal velocity \( v^* \), 
both non-dimensionalized with respect to the freestream velocity \( u_\infty \). The streamwise velocity starts at zero at the wall (due to the no-slip condition) and increases 
smoothly with \( \eta \), approaching the freestream value of 1 as \( \eta \) reaches around 5. This behavior represents the transition from the boundary layer to the freestream. 
The wall-normal velocity starts at zero at the wall and remains small throughout the boundary layer reflecting the slight upward movement of fluid near the wall due to the boundary layer growth, but it does not show a large peak

\subsection*{Part (d): Significance of f''(0)}
The quantity \( f''(0) \), which is the second derivative of \( f \) with respect to \( \eta \) at \( \eta = 0 \), is significant
because it is directly related to the shear stress at the wall as it represents the velocity gradient at the wall
This velocity gradient determines how much shear is exerted by the fluid on the wall. 

\begin{thebibliography}{9}
    \bibitem{GitHubRepo}
    \textit{CFD Repository},\\
    Available at: \url{https://github.com/GiuseppePisante/CFD.git}
    
    \bibitem{GitHubCopilot}
    \textit{GitHub Copilot},\\
    GitHub. Available at: \url{https://github.com/features/copilot}

    \bibitem{clarkrichards2022} Clark Richards, 
    \textit{Solving the Blasius Equation},\\
    Avaible at: \url{https://www.clarkrichards.org/2022/06/22/solving-the-blasius-equation/}.
    \end{thebibliography}
       
\end{document}