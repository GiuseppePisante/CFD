\documentclass{article}
\usepackage{amsmath}
\usepackage{titlesec}
\usepackage{graphicx}
\usepackage[margin=1in]{geometry}
\usepackage{hyperref}

% Title, date, and author
\title{Exercise 1}
\author{Your Name, Collaborator's Name}
\date{\today}

\titleformat{\section}
  {\normalfont\normalsize\bfseries} % Format: font style, size, and weight
  {\thesection}{1em} % Label format and spacing
  {}
  \renewcommand{\thesubsection}{\thesection.\alph{subsection}}

\titleformat{\subsection}
  {\normalfont\small\bfseries} % Format: font style, size, and weight
  {\thesubsection}{1em} % Label format and spacing
  {}
\titleformat{\subsubsection}
  {\normalfont\small\bfseries} % Format: font style, size, and weight
  {\thesubsubsection}{1em} % Label format and spacing
  {}

\begin{document}
\begin{titlepage}
    \centering
    \vspace*{1in}
    
    {\Huge\bfseries Exercise 1\par}
    \vspace{1.5cm}
    {\Large \today\par}
    \vspace{1.5cm}
    {\Large\itshape Antonio Pampalone 23586519 \\ Giuseppe Pisante 23610012\\ Martina Raffaelli 23616907 \par}
    
    \vfill
    \includegraphics[width=0.3\textwidth]{FAU-Logo.png}\par\vspace{1cm} % Adjust the width as needed
   
\end{titlepage}

\newpage
\small

\section*{Task 2.2: Order Reduction}

The governing equation of the damped oscillator is given by:
\begin{equation}
m \frac{d^2 y(t)}{dt^2} + b \frac{dy(t)}{dt} + c y(t) = 0
\end{equation}

with initial conditions:
\[
y(0) = s_0, \quad \frac{dy(0)}{dt} = v_0.
\]

We aim to transform this second-order ODE into a system of first-order differential equations. To reduce a second-order ODE to a 
system of first-order ODEs we can introduce new variables to represent the derivatives of the function \( y(t) \). 
In particular, we define:
\[
y_1(t) = y(t)
\]

and introduce a new variable \( y_2(t) \) to represent the first derivative of \( y(t) \):
\[
y_2(t) = \frac{dy(t)}{dt}.
\]

Since \( \frac{dy_2(t)}{dt} = \frac{d^2 y(t)}{dt^2} \), we can substitute this into the original equation to obtain:
\begin{equation}
m \frac{dy_2(t)}{dt} + b y_2(t) + c y_1(t) = 0.
\end{equation}

In this way, we can express the problem as two coupled first-order differential equations:
\[
\begin{cases}
\frac{dy_1(t)}{dt} &= y_2(t), \\
\frac{dy_2(t)}{dt} &= -\frac{b}{m} y_2(t) - \frac{c}{m} y_1(t).
\end{cases}
\]

However, we also need to rewrite the initial conditions for \( y(t) \) and \( \frac{dy(t)}{dt} \):
\begin{itemize}
    \item \( y_1(0) = s_0 \),
    \item \( y_2(0) = v_0 \).
\end{itemize}

This approach allows us to solve the system using methods suited for first-order differential equations, enabling easier numerical or analytical analysis.










\section*{Task 2.3: Blasius Equation}

\subsection*{Part (a): Convert the Blasius Equation to a System of First-Order ODEs}

The Blasius equation is given by:

\begin{equation}
f''' + \frac{1}{2} f f'' = 0
\end{equation}

with \( f' = \frac{u}{U_\infty} \). Three boundary conditions are necessary to solve this equation:
\begin{itemize}
    \item \( \eta = 0 \): \( f' = f = 0 \) (no-slip condition)
    \item \( \eta \to \infty \): \( f' = 1 \) (free outer flow)
\end{itemize}

We aim to transform this third-order ODE into a system of first-order differential equations. To reduce a third-order ODE to a 
system of first-order ODEs we can introduce new variables to represent the derivatives of the function \( f(\eta) \). 
In particular, we define:
\[
y_1 = f, \quad y_2 = f' = \frac{df}{d\eta}, \quad y_3 = f'' = \frac{d^2 f}{d\eta^2}
\]

Then, the derivatives of these variables with respect to \(\eta\) are:
\[
\frac{dy_1}{d\eta} = y_2, \quad \frac{dy_2}{d\eta} = y_3
\]

Now, we can substitute this into the original Blasius equation to obtain:
\[
\frac{dy_3}{d\eta} = -\frac{1}{2} y_1 y_3
\]

In this way, we can express the problem as three coupled first-order differential equations:
\[
\begin{cases}
\frac{dy_1}{d\eta} = y_2 \\
\frac{dy_2}{d\eta} = y_3 \\
\frac{dy_3}{d\eta} = -\frac{1}{2} y_1 y_3
\end{cases}
\]

with boundary conditions:
\begin{itemize}
    \item At \( \eta = 0 \): \( y_1 = 0 \), \( y_2 = 0 \)
    \item As \( \eta \to \infty \): \( y_2 = 1 \)
\end{itemize}

\subsection*{Part (b): Providing an Initial Condition for \( f''(0)\)}

To solve this problem as an initial value problem, we need an initial value for \( f''(0) \). However, the boundary condition \( y_2(\infty) = 1 \) is 
specified at infinity, making it impractical to impose this condition directly at a finite point. To address this, we can use an iterative approach:

\begin{enumerate}
    \item Guess an initial value for \( f''(0) \).
    \item Integrate the system of equations from \( \eta = 0 \) to a sufficiently large value of \( \eta \)  where \( y_2(\eta) \) approaches a constant. 
    To solve this system numerically, we use the Runge-Kutta method of fourth order (RK4) that allows to approximate solutions to ordinary differential equations.
    
    For a step size \( h \), the RK4 method computes the next values \( y_{i+1} \) as follows:
    
    \[
    k_1 = h \cdot f(\eta_i, y_i)
    \]
    \[
    k_2 = h \cdot f\left(\eta_i + \frac{h}{2}, y_i + \frac{k_1}{2}\right)
    \]
    \[
    k_3 = h \cdot f\left(\eta_i + \frac{h}{2}, y_i + \frac{k_2}{2}\right)
    \]
    \[
    k_4 = h \cdot f(\eta_i + h, y_i + k_3)
    \]
    
    The next value of the solution is updated by:
    
    \[
    y_{i+1} = y_i + \frac{1}{6} \left(k_1 + 2k_2 + 2k_3 + k_4\right)
    \]

    \item Check if \( y_2(\eta) \) approaches 1 as \( \eta \to \infty \). If \( y_2(\eta) \) is not close to 1, adjust the initial guess for \( y_3(0) \) 
    iterate this process until the condition \( y_2(\infty) = 1 \) (or close to it) is satisfied within a desired tolerance.
\end{enumerate}

This iterative approach allows us to find an appropriate initial condition for \( y_3(0) = f''(0) \) that satisfies the boundary condition at infinity.

\end{document}
