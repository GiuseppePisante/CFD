\documentclass{article}
\usepackage{amsmath}
\usepackage{titlesec}
\usepackage{graphicx}
\usepackage[margin=1in]{geometry}
\usepackage{hyperref}
\usepackage{float}

% Title, date, and author
\title{Exercise 5}
\author{Your Name, Collaborator's Name}
\date{\today}

\titleformat{\section}
  {\normalfont\normalsize\bfseries} % Format: font style, size, and weight
  {\thesection}{1em} % Label format and spacing
  {}
  \renewcommand{\thesubsection}{\thesection.\alph{subsection}}

\titleformat{\subsection}
  {\normalfont\small\bfseries} % Format: font style, size, and weight
  {\thesubsection}{1em} % Label format and spacing
  {}
\titleformat{\subsubsection}
  {\normalfont\small\bfseries} % Format: font style, size, and weight
  {\thesubsubsection}{1em} % Label format and spacing
  {}

\begin{document}
\begin{titlepage}
    \centering
    \vspace*{1in}
    
    {\Huge\bfseries Exercise 5\par}
    \vspace{1.5cm}
    {\Large \today\par}
    \vspace{1.5cm}
    {\Large\itshape Antonio Pampalone 23586519 \\ Giuseppe Pisante 23610012\\ Martina Raffaelli 23616907 \par}
    
    \vfill
    \includegraphics[width=0.3\textwidth]{FAU-Logo.png}\par\vspace{1cm} % Adjust the width as needed
   
\end{titlepage}

\newpage
\small
\section{Implicit Euler scheme for diffusion equation}
\subsection{Discretization}
Recalling the general form of the implicit Euler method:
\begin{equation*}
  \phi^{n+1}_{i,j} = \phi^{n}_{i,j} + f(\phi^{n+1}, t^{n+1}) \Delta t 
\end{equation*}
we get for the diffusion equation:
\begin{equation*}
  \phi^{n+1}_{i,j} = \phi^{n}_{i,j} + \alpha (\frac{\partial^2 \phi^{n+1}_{i,j}}{\partial x^2} + \frac{\partial^2 \phi^{n+1}_{i,j}}{\partial y^2}) \Delta t
\end{equation*}

Then we use the second order central difference scheme for the spatial derivatives:
\begin{equation*}
  \frac{\partial^2 \phi^{n+1}_{i,j}}{\partial x^2} = \frac{\phi^{n+1}_{i+1,j} - 2\phi^{n+1}_{i,j} + \phi^{n+1}_{i-1,j}}{\Delta x^2}
\end{equation*}
\begin{equation*}
  \frac{\partial^2 \phi^{n+1}_{i,j}}{\partial y^2} = \frac{\phi^{n+1}_{i,j+1} - 2\phi^{n+1}_{i,j} + \phi^{n+1}_{i,j-1}}{\Delta y^2}
\end{equation*}

Substituting the above equations into the diffusion equation, we get:
\begin{equation} \label{discretization}
  \phi^{n+1}_{i,j} = \phi^{n}_{i,j} + \alpha \left( \frac{\phi^{n+1}_{i+1,j} - 2\phi^{n+1}_{i,j} + \phi^{n+1}_{i-1,j}}{\Delta x^2} + \frac{\phi^{n+1}_{i,j+1} - 2\phi^{n+1}_{i,j} + \phi^{n+1}_{i,j-1}}{\Delta y^2} \right) \Delta t
\end{equation}
where we can rearrange the terms to get the following expression:
\begin{equation}
  \phi^{n+1}_{i,j} (1 + 2 \alpha \Delta t(\frac{1}{\Delta x^2} + \frac{1}{\Delta y^2})) - \alpha \Delta t \left( \frac{\phi^{n+1}_{i+1,j} + \phi^{n+1}_{i-1,j}}{\Delta x^2} + \frac{\phi^{n+1}_{i,j+1} + \phi^{n+1}_{i,j-1}}{\Delta y^2} \right) = \phi^{n}_{i,j}
\end{equation}

\subsection{Consistency proof}
In order to prove consistency of the discretization we have to show that the truncation error $T$ goes to zero as the grid spacing ($\Delta x, \Delta y$) goes to zero and the time step ($\Delta t$) goes to zero.


We start with the exact form of the diffusion equation:

\begin{equation}
\frac{\partial \phi}{\partial t} = \alpha \left( \frac{\partial^2 \phi}{\partial x^2} + \frac{\partial^2 \phi}{\partial y^2} \right).
\end{equation}

The discretized scheme for the implicit Euler method is described by equation \eqref{discretization}.

To analyze the truncation error, we expand \( \phi_{i+1,j}^{n+1}, \phi_{i-1,j}^{n+1}, \phi_{i,j+1}^{n+1}, \phi_{i,j-1}^{n+1} \) using Taylor series around \( \phi_{i,j}^{n+1} \). For example, the expansion for \( \phi_{i+1, j}^{n+1} \) is given by:

\begin{equation*}
\phi_{i+1,j}^{n+1} = \phi_{i,j}^{n+1} + \Delta x \frac{\partial \phi}{\partial x} + \frac{\Delta x^2}{2} \frac{\partial^2 \phi}{\partial x^2} + \frac{\Delta x^3}{6} \frac{\partial^3 \phi}{\partial x^3} + \cdots,
\end{equation*}

and similar expansions hold for the other terms. Then we substitute these expansions into the discretized scheme. For example, the term 

\begin{equation*}
\frac{\phi_{i+1, j}^{n+1} - 2\phi_{i,j}^{n+1} + \phi_{i-1,j}^{n+1}}{\Delta x^2}
\end{equation*}

becomes:

\begin{equation*}
\frac{\partial^2 \phi}{\partial x^2} + \frac{\Delta x^2}{12} \frac{\partial^4 \phi}{\partial x^4} + \cdots.
\end{equation*}

Substituting all the Taylor expansions into the discretized equation and rearranging terms, we obtain the residual or truncation error as:

\begin{equation}
T = \Delta t \frac{\partial^2 \phi}{\partial t^2} + \frac{\Delta x^2}{12} \frac{\partial^4 \phi}{\partial x^4} + \frac{\Delta y^2}{12} \frac{\partial^4 \phi}{\partial y^4} + \cdots.
\end{equation}

As \( \Delta t \to 0 \), \( \Delta x \to 0 \), and \( \Delta y \to 0 \), the truncation error \( T \) approaches zero, demonstrating that the discretization is consistent.


\subsection{Stability criteria}

To determine the stability criterion for the implicit Euler scheme using the Von Neumann method, we start with the discretized equation \eqref{discretization} and we
introduce the parameters $ r_x = \frac{\alpha \Delta t}{\Delta x^2} $ and $ r_y = \frac{\alpha \Delta t}{\Delta y^2} $, then the equation becomes:
\begin{equation*}
\phi_{i,j}^{n+1} = \phi_{i,j}^n + r_x \left( \phi_{i+1,j}^{n+1} - 2\phi_{i,j}^{n+1} + \phi_{i-1,j}^{n+1} \right) + r_y \left( \phi_{i,j+1}^{n+1} - 2\phi_{i,j}^{n+1} + \phi_{i,j-1}^{n+1} \right).
\end{equation*}

Rearranging terms, we write:
\begin{equation*}
\phi_{i,j}^{n+1} \left( 1 + 2r_x + 2r_y \right) = \phi_{i,j}^n + r_x \left( \phi_{i+1,j}^{n+1} + \phi_{i-1,j}^{n+1} \right) + r_y \left( \phi_{i,j+1}^{n+1} + \phi_{i,j-1}^{n+1} \right).
\end{equation*}

The Von Neumann method assumes a Fourier mode solution of the form:
\begin{equation*}
\phi_{i,j}^n = G^n e^{I(k_x i \Delta x + k_y j \Delta y)},
\end{equation*}
where $ G $ is the amplification factor,$ k_x $ and $ k_y $ are the wave numbers in the $ x $- and $ y $-directions, and $ I = \sqrt{-1} $.

We substituting this Fourier mode into the discretized equation, and the neighboring terms become:
\begin{equation*}
\phi_{i+1,j}^{n+1} = G e^{I k_x \Delta x} \phi_{i,j}^{n+1}, \quad \phi_{i-1,j}^{n+1} = G e^{-I k_x \Delta x} \phi_{i,j}^{n+1},
\end{equation*}
\begin{equation*}
\phi_{i,j+1}^{n+1} = G e^{I k_y \Delta y} \phi_{i,j}^{n+1}, \quad \phi_{i,j-1}^{n+1} = G e^{-I k_y \Delta y} \phi_{i,j}^{n+1}.
\end{equation*}

Using these expressions, substitute into the equation. Simplify the exponentials using $ e^{I \theta} + e^{-I \theta} = 2 \cos \theta $, leading to:
\begin{equation*}
G \phi_{i,j}^{n+1} \left( 1 + 2r_x + 2r_y \right) = \phi_{i,j}^n + 2r_x G \phi_{i,j}^{n+1} \cos(k_x \Delta x) + 2r_y G \phi_{i,j}^{n+1} \cos(k_y \Delta y).
\end{equation*}

Factoring $G$ from the left-hand side:

\begin{equation*}
G \left( 1 + 2r_x + 2r_y - 2r_x \cos(k_x \Delta x) - 2r_y \cos(k_y \Delta y) \right) = 1.
\end{equation*}

We solve for $ G $:
\begin{equation*}
G = \frac{1}{1 + 2r_x + 2r_y - 2r_x \cos(k_x \Delta x) - 2r_y \cos(k_y \Delta y)}.
\end{equation*}

The scheme is stable if $ |G| \leq 1 $. For the implicit Euler method, the denominator in $ G $ is always greater than 1 for all $ k_x $ and $ k_y $. So we have that:
\begin{equation}
|G| \leq 1,
\end{equation}
which implies unconditional stability.

\subsection{Convergence proof}
Since the scheme is both consistent and stable, for the Lax equivalence theorem, the scheme is convergent.


\begin{thebibliography}{9}
  \bibitem{GitHubRepo}
  \textit{CFD Repository},\\
  Available at: \url{https://github.com/GiuseppePisante/CFD.git}
  
  \bibitem{GitHubCopilot}
  \textit{GitHub Copilot},\\
  GitHub. Available at: \url{https://github.com/features/copilot}
  
  \bibitem{HeatEquation}
  MIT OpenCourseWare,\\
  \textit{Heat Equation Notes}, 2006,\\
  Available at: \url{https://ocw.mit.edu/courses/18-303-linear-partial-differential-equations-fall-2006/d11b374a85c3fde55ec971fe587f8a50_heateqni.pdf},\\
\end{thebibliography}

\end{document}